\documentclass[a4paper]{article}
\usepackage[utf8]{inputenc}
\usepackage{fullpage}
\usepackage{csquotes}
\usepackage[ngerman]{babel}
\usepackage{float}
\usepackage{graphicx}
\usepackage{epstopdf}
\usepackage{subfigure}
\setcounter{secnumdepth}{-1} 
\usepackage{hyperref}
\usepackage{listings}
\usepackage{dsfont}
\title{Betriebssysteme Vertiefung \\ Übung 5}

\author{Dominik Eckelmann, Matr.-Nr.: 785856}
\date{\today}

\lstset{
language=C,                % the language of the code
numbers=left,                   % where to put the line-numbers
numbersep=5pt,                  % how far the line-numbers are from the code
showspaces=false,               % show spaces adding particular underscores
showstringspaces=false,         % underline spaces within strings
showtabs=false,                 % show tabs within strings adding particular underscores
frame=single,                   % adds a frame around the code
tabsize=2,                      % sets default tabsize to 2 spaces
breaklines=true,                % sets automatic line breaking
breakatwhitespace=false        % sets if automatic breaks should only happen at whitespace
}

\begin{document}

\maketitle

\tableofcontents

\section{Einleitung}
Als Teil der Lehrveranstaltung \textit{Betriebssysteme Vertiefung} im Wintersemester 2011/2012 an der \textit{Beuth Hochschule für Technik Berlin} sollte im Rahmen einer Übung mithilfe der Programmiersprache C ein Programm geschrieben werden, eine Datei kopiert. Hierbei soll ein Thread zum lesen der Datei, sowie ein Thread zum schreiben der Datei verwendet werden. Zudem soll ein Ringbuffer verwendet werden.

\section{Problembeschreibung}
Der Aufgabe liegt das Producer-Consumer Problem zugrunde.
Zunächst einmal hat der Ringbuffer eine bestimmte Kapazität.
Er wird vom produzierenden Thread mit Daten gefüllt.
Der konsumierende Thread liest die Daten aus dem Buffer.
Sobald einer der Threads das Ende erreicht, setzt er seine Arbeit
am Anfang des Threads fort.

Hierbei ist es wichtig darauf zu achten, dass der Produzierende Thread zunächst
immer vor dem dem konsumierenden Thread startet und das die beiden Threads sich nicht
überholen dürfen.

\section{Lösungen}

Zur Lösung in C werden zwei Semaphoren verwendet. 
Eine Semaphore \texttt{write}, welche mit der größe des Ringbuffers abzüglich eins initialisiert wird. Die andere Semaphore \texttt{read}, mit der initialgröße 0.

Die Semaphoren werden nun folgendermaßen verwendet:
\begin{verbatim}
Producer-Thread:
    P(write)
    Ringbuffer befüllen
    V(read)
Consumer-Thread:
    P(read)
    Aus Ringbuffer lesen
    V(write)
\end{verbatim}

Wie klar ersichtlich blockieren sich die beiden Threads selbst und heben die Sperre je für den
anderen auf. Auf diese Weiße wird sichergestellt das sich die beiden Threads nicht überholen.

\end{document}